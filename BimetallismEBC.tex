
\chapter{Bimetallism in Bitcoin and Litecoin?}

The situation of multiple currencies, especially in light of the gold
analogy, brings to mind the idea of bimetallism, the monetary system
where the currency is based not just on gold, but also on silver.
The two being interchangeable at a fixed rate.

Also in the light of bimetallism, we can consider Gresham's law. Bad
money drives out good. In the case of bimetallism, the most circulated
specie will be the one which is currently considered to be undervalued
by the official exchange rate. Meaning that if silver becomes more
abundant, its relative value will decrease, and people will first
spend their silver coins, before they spend their gold ones. Since
exchange rates between cryptocurrencies are freely floating, there
is no 'bad money', since each currency is valued at market rate. Interestingly
enough, the rate between Bitcoin and Litecoin seems to trade in a
very narrow band, especially considering the volatility vis-a-vis
fiat currencies. 

However, there is a crowding out of sort. This is perhaps best compared
to international currency competition. Where there is a competition
for the privilege of being a reserve currency. A foreign investor
can choose exchange his national fiat currency for e.g. the US Dollar
or the Euro. In these markets, the good currencies drive out the bad.
