
\chapter{Is Bitcoin Money?}

What is money? Sometimes it is obvious, well established currencies
such as the US Dollar, the Japanse Yen or the Swiss Franc are. How
about the Euro, it is not a national currency, it is not owned by
a state, but rather by a collective of member state representatives.
Zimbabwe owns their own currency, but this has become essentially
useless. What if money is moved from one, bank account to another
bank account at the same bank in the same country, through ebanking.
No physical change will occur (save for a few bits on a computer driver
somewhere in some remote datacentre). In some places sea shells or
other rare commodities such a gold or ivory are accepted. Just as
the number of possible moneys is alsmost infiniate, so is the number
of definitions, but a comptemporary (pre-Bitcoin) consensus would
dictate somethink like: 
\begin{quotation}
an object or entry in a record that is generally accepted as payment
for goods and services and debts in a certain socio-economic context
or country \textcolor{red}{CITE!!}.
\end{quotation}
We can observe a number of things from this definition. Firstly, money
does not have to be an object (such as a coin), but can also be an
entry in a record (such as a bank account balance). Secondly, it assumed
to be generally accepted (though not universally). Thirdly, a money
is not limited to only being accepted in a country, but can also be
accepted in a certain context. Lastly, the existance of one money
does not rule out the existance of others.

Bitcoin, the blockchain public ledger is a record. Generally accepted,
but generally accepted within a certain context. Also, coexcistance
of moneys.

The question of whether Bitcoin is money is a complicated question,
there are some standard which are used in academia to evaluate whether
something is money or not. Besides this there are the issues of legal
tender, and legal recognition. We will start by discussion the academic
criteria for a money, after which we will look at legal tender and
legal recognition. The standard criteria used in academia for analysing
money are:
\begin{enumerate}
\item medium of exchange
\item unit of account or numéraire
\item store of value
\end{enumerate}
These criteria can broadly be interpreted as follow. A medium of exchange

The criteria (1.) and (3.) are clearly fulfilled. Since cryptocoins
are virtual goods, and multiple copies can be made, there is no degeneration
over time. Furthermore, cryptocoins are a good medium of exchange,
transactions are quick and relatively cost less. The unit of account
is not a criterium that is currently well met. The volatility of cryptocurrencies
vis--vis fiat currencies makes it difficult to denominate goods or
services in these in cryptocoins. However, in that sense it is again
similar to gold, which also has a highly volatile price as denoted
in fiat currency.


\section*{Legal Tender}

In discussion about the status of Bitcoin as money, the issue of legal
tender is often heard. The discussion is somewhat misguided in the
sense that legal tender is more of a means to an end, rather than
a goal in its own right.

Legal tender is a rule in which a government imposes on the inhabitants
of a territory, that within that territory everyone must accept the
currency deemed legal tender, as payment for debt, goods, or services.
Legal tender is thus a way for a government to create a market for
a currency, this currency generally being the one controlled by the
same government. There are many good reasons for doing so. It creates
a certain level of certainty, if I want to buy something in Switzerland,
I know that Swiss Francs will be a means of paying. In short, if somebody
offers to pay in a currency that is legal tender in the relevant territory,
the seller has to accept this.

At the time of writing, Bitcoin has not been deemed legal tender anywhere,
so how problematic is this? As mentioned above, legal tender is means
for government to create a market for a currency. Bitcoin seeks to
establish itself through a popular mandate, rather than a government
one. That is to say, by being a more competitive means of transaction
for both the retailer and the customer. If Bitcoin is more favorable
to both customer and retailer, then the customer can offer to pay
in Bitcoin, and the retailer can choose to accept it. In it thus through
being more interesting as a method of payment, rather than through
a government mandate that Bitcoin seems to be accepted.
