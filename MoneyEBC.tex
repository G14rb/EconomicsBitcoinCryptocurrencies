
\chapter{Is Bitcoin money?}

What is money? Some answers are obvious, well established currencies
such as the US Dollar are. But the Euro is not a national currency,
it is not owned by a state. Zimbabwe owns their own currency, but
this has become essentially useless. What if money is moved from one,
bank account to another bank account at the same bank in the same
country, through ebanking. No physical change will occur (save for
a few bits on a computer driver somewhere in some remote datacentre).
In some places sea shells or other rare commodities such a gold or
ivory are accepted. Just as the number of possible moneys is alsmost
infiniate, so is the number of definitions, but a comptemporary (pre-Bitcoin)
consensus would dictate somethink like: 
\begin{quotation}
an object or entry in a record that is generally accepted as payment
for goods and services and debts in a certain socio-economic context
or country \textcolor{red}{CITE!!}.
\end{quotation}
We can observe a number of things from this definition. Firstly, money
does not have to be an object (such as a coin), but can also be an
entry in a record (such as a bank account balance). Secondly, it assumed
to be generally accepted (though not universally). Thirdly, a money
is not limited to only being accepted in a country, but can also be
accepted in a certain context. Lastly, the existance of one money
does not rule out the existance of others.

Bitcoin, the blockchain public ledger is a record. Generally accepted,
but generally accepted within a certain context. Also, coexcistance
of moneys.

The question of whether Bitcoin is money is a complicated question,
there are some standard which are used in academia to evaluate whether
something is money or not. Besides this there are the issues of legal
tender, and legal recognition. We will start by discussion the academic
criteria for a money, after which we will look at legal tender and
legal recognition. The standard criteria used in academia for analysing
money are:
\begin{enumerate}
\item medium of exchange
\item unit of account or numéraire
\item store of value
\end{enumerate}
These criteria can broadly be interpreted as follow. A medium of exchange

The criteria (1.) and (3.) are clearly fulfilled. Since cryptocoins
are virtual goods, and multiple copies can be made, there is no degeneration
over time. Furthermore, cryptocoins are a good medium of exchange,
transactions are quick and relatively cost less. The unit of account
is not a criterium that is currently well met. The volatility of cryptocurrencies
vis--vis fiat currencies makes it difficult to denominate goods or
services in these in cryptocoins. However, in that sense it is again
similar to gold, which also has a highly volatile price as denoted
in fiat currency.


\section*{Legal Issues}

As mentioned above, there are two legal issues, recognition as being
legal, alongside a legal classification of Bitcoin, as well as the
issue of legal tender. We will discuss these here.


\subsection*{Legal Recognition and Classification}

In the initial days of Bitcoin, especially when it was mainly used
for purchasing illicit goods, some held the belief that most governments
would perceive Bitcoin as a threat to their monopoly on currency and
for this reason would ban it. This already being contrasted by governments
of the EMU member states, who have volutarily surrendered their monetary
monopoly for a 'foreign' currency. It turned out most government where
more interested in taxing it. For this reason Bitcoin has in many
places been classified as a financial instument. If Bitcoins are derived
from mining, they are to be taxed using income tax. If Bitcoins are
purchased these will be taxed as financial holdings.


\subsection*{Legal Tender}

What is legal tender? why do we need it, why dont we need it in bitcoin,
refer to the chapter of why bictoin
