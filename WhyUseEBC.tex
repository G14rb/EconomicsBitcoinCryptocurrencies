
\chapter{Why use Bitcoin?}

Having examined the technological side of \nomenclature{Bitcoin}{  },
we can now consider some situations in which these innovations best
come to fruitition. In examining these situations we consider these
merely from an effectiveness perspective, without regard for the morality
of such possible situations, which is beyond the scope of this book.
\begin{itemize}
\item international remittances
\item online transactions
\item mobile payments
\item evasion of capital controls
\item evasion of taxation
\item anonymous online transactions
\end{itemize}
This makes \nomenclature{Bitcoin}{  } an ideal mechanism for remittances,
which nowadays can cost as much as 10 percent. Additionally no government
can apply capital restrictions on these remittances. 

The instant nature of transactions means it is well suited for online
retailing. Credit card transactions seem instant, but in fact are
not. A significant percentage is in fact reversed after initial clearance.
This is very costly for retailers, this often occurs after packaging
of shipping has started.
