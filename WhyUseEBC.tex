
\chapter{Why use Bitcoin?}

There are some key benefits to cryptocurrencies such as Bitcoin, a
few of which I will mention here.

Firstly, the system is self-regulating, meaning that no government
or institution controls it. For the user that means that their holdings
cannot be used as a policy instrument. Conventional fiat currencies
are owned by a government, which means that they can be used to e.g.
stimulate economic growth by printing extra money, which lessens the
value of individuals holdings.

Secondly, users control their own holdings on their own computer,
which eliminates the need for a bank. This brings the advantage that
users are not dependent on open hours, waiting lines, or e-banking
websites, it also means no fees. Additionally a user's holdings cannot
be wiped out by his or her bank bankrupting after speculative investing
or something of that sort.

Thirdly, transactions are as easy, instant, and costless as sending
an email. Bitcoin exists online digitally, which means that there
is no physical counterpart that needs to be moved, such as with gold,
or paper money. Additionally, it is equally valid across countries,
which means that there is no need to exchange it. This makes Bitcoin
an ideal mechanism for remittances, which nowadays can cost as much
as 10 percent. Additionally no government can apply capital restrictions
on these remittances.

Lastly, the instant nature of transactions means it is well suited
for online retailing. Credit card transactions seem instant, but in
fact are not. A significant percentage is in fact reversed after initial
clearance. This is very costly for retailers, this often occurs after
packaging of shipping has started.
