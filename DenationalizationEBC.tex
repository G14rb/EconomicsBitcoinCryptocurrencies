
\chapter{Cryptocurrency competition: Hayek's Paradise}

Bitcoin first become popular in several underground movements. As
such it has been hauled as a major innovation in a self-identified
movement know as \nomenclature{crypto-anarchism}{ }. The main idea
behind crypto-anarchism is that cryptography empowers individuals
to avoid government control. It is not suprising that many of the
economic principles on which \nomenclature{Bitcoin}{ } is based,
are commonly identified with government weary schools of thought is
economics academia, such as the Austrian School and Monetary Economics. 

In a \textcolor{red}{1999 interview}, Milton Friedman, the key proponent
of Monetary Economics gave an interview to the tax-payers association
of America in which he says:
\begin{quotation}
\textcolor{red}{The one thing that is missing -but will be invented
very soon- is a reliable anonymous online e-cash}
\end{quotation}
If we consider the cryptocurrency space as a whole we seem to approach
the 'Decentralization of Money' as descibed by \textcolor{red}{Friedrich
Hayek (1976)}. Hayek describes a situation in which currencies are
no longer owned by governments, but in stead are owned by private
companies. In this transnational currency space, the most currency
or currencies most favorable to consumers would be adopted strongest
and would become dominant. The situation of there being several competing
cryptocurrencies in comes very close to the system as invisioned by
Hayek. It also addresses some of the key critisisms of this vision,
as raised for example by Milton Friedman...

\textcolor{red}{PERHAPS A DISCUSSION OF GRESHAMS LAW IN THIS CRYPTOCURRENCY
COMPETITION}

Although Hayek firmly believed in the self-regulating currency market,
the market itself was made up of self-serving firms which owned and
controlled the currencies. Cryptocurrencies go a step further, in
making the currencies self-regulating and making only the individuals
currency holders self-serving actors.
