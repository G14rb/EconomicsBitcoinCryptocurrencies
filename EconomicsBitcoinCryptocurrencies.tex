%% LyX 2.1.0rc1 created this file.  For more info, see http://www.lyx.org/.
%% Do not edit unless you really know what you are doing.
\documentclass[oneside,british,a5paper]{book}
\usepackage[T1]{fontenc}
\usepackage[utf8]{luainputenc}
\setcounter{secnumdepth}{3}
\setcounter{tocdepth}{3}
\usepackage{color}
\usepackage{babel}
\usepackage{nomencl}
% the following is useful when we have the old nomencl.sty package
\providecommand{\printnomenclature}{\printglossary}
\providecommand{\makenomenclature}{\makeglossary}
\makenomenclature
\usepackage[unicode=true,
 bookmarks=true,bookmarksnumbered=false,bookmarksopen=false,
 breaklinks=false,pdfborder={0 0 0},backref=false,colorlinks=false]
 {hyperref}
\hypersetup{pdftitle={Economics of Bitcoin and Cryptocurrenies},
 pdfauthor={Bastiaan Quast},
 pdfkeywords={bitcoin, cryptocurrencies, economics}}

\makeatletter
%%%%%%%%%%%%%%%%%%%%%%%%%%%%%% Textclass specific LaTeX commands.
\newcommand{\lyxaddress}[1]{
\par {\raggedright #1
\vspace{1.4em}
\noindent\par}
}

%%%%%%%%%%%%%%%%%%%%%%%%%%%%%% User specified LaTeX commands.
\usepackage[authordate,natbib,backend=biber]{biblatex-chicago}

\usepackage[toc]{glossaries}
\glossarystyle{altlistgroup}

% add the bibliography file (linux location)
\addbibresource{~/EconomicsBitcoinCryptocurrencies/bibliography.bib} 

\newglossaryentry{Bitcoin-Qt}{name={Bitcoin-Qt},description={The original Bitcoin software, it is still used, but many other packages have also been developed.}}

\newglossaryentry{Bitcoin}{name={Bitcoin},description={The original cryptocurrency}}

\newglossaryentry{bitcoin}{name={bitcoin},description={the unit of account in the Bitcoin system}}

\newglossaryentry{open source}{name={open source},description={Software of which the source code is publicly available for inspection}}

\newglossaryentry{satoshi}{name={satoshi},description={The atomic unit of account in the Bitcoin system, it is equal to 1/1,00,000,000 bitcoin, named after the Bitcoin creator Satoshi Nakamoto}}

\newglossaryentry{Satoshi Nakamoto}{name={Satoshi Nakamoto},description={The pseudonym of the original Bitcoin inventor and creator}}

\newglossaryentry{cryptocurrency}{name={cryptocurrency},description={the generic term for online, cryptography based currencies, Bitcoin is the original cryptocurrency}}

\newglossaryentry{blockchain}{name={blockchain},description={the public ledget of the Bitcoin system, it contains all transactions ever recorded}}

\newglossaryentry{wallet}{name={wallet},description={the collection of private and public codes used to commit transactions}}

\newglossaryentry{miner}{name={miner},description={a computer which participates in the Bitcoin transaction verification process}}

\newglossaryentry{signature}{name={signature},description={a digital code used to verify that the originator of a transaction owns the sending address}}

\newglossaryentry{private key}{name={private key},description={the code which can be used to commit transactions}}


\makeglossaries

\let\nomenclature\gls

\makeatother

\begin{document}
\frontmatter


\title{The Economics of Bitcoin and Cryptocurrencies}


\author{Bastiaan Quast}

\maketitle

\lyxaddress{The Graduate Institute}


\lyxaddress{Mainson de la Paix}


\lyxaddress{Chemin de Eugene Rigot 2}


\lyxaddress{Case Postal 136, 1211 Geneve 21}


\lyxaddress{Geneva}


\lyxaddress{Switzerland}


\chapter*{About the author}

Bastiaan Quast is a PhD Candidate in Development Economics at The
Graduate Institute of International and Development Studies in Geneva.

He holds a Master in Quantitative Economics and Finance from the University
of St. Gallen in Switzerland as well as a Bachelor in Theoretical
Philosophy and a Bachelor in Economics from the University of Groningen
in The Netherlands.

\tableofcontents{}

\begin{center}
\mainmatter
\par\end{center}


\chapter{Introduction}

In 2009 the online currency\textbf{ \nomenclature{Bitcoin}{ }} was
launched by an anonymous developer known by the pseudonym \nomenclature{Satoshi Nakamoto}{ }.
Initially the project remained fairly low key, have only a small group
of geeks participate in it. However, in 2013 Bitcoin quickly became
a mainstream phenomenon as it was reported on in the mainstream media
such as The Economics, Time Magaine, The New York Times, and many
more. This increased attention created a surge in the uptake of Bitcoin
both as an investment as well as transaction mechanism. However, Bitcoin
also has a dark side, being initially used mostly to purchase illicit
goods on online marketplaces such The Silk Road, websites that are
generally only available through the dark web. However, the rise in
popular attention has also brought the attention of regulators, which
has caused many of the illegal marketplaces to be taken offline, as
well as introduced a range of fiscal regulation, dealing with bitcoin.
This book provides a laymens introduction to the basic principles
of Bitcoin and other online currencies, as well as discuss how this
new phenomenon can be viewed from an economic perspective, and what
that would mean for the development of such systems.

The Bitcoin system was based on a 2008 white paper by the same author
\citep{nakamoto2008bitcoin}. 

Besides the fact that Bitcoin exist only digitally, its defining characteristics
are decentralisation and openness. and every aspect of it is public
and \nomenclature{open source}{ }. 

This quickly led to the launching of several alternative online currencies,
based on the Bitcoin protocol. A currency based on cryptography, such
as Bitcoin, is know as a \nomenclature{cryptocurrency}{ }.

Cryptocurrencies come with an extensive new terminology, for reference,
a glossary is provided in the back of the book.


\part{Basic principles}


\chapter{What is Bitcoin?}

A short answer is that it is an online anonymous currency which is
not controlled by any single government or entity. However, from a
technical perspective there are many innovative features to Bitcoin,
and I will begin by highlighting the key ones. 

\nomenclature{Bitcoin}{ }is a confusing term. It is both a monetary
unit of denomination, such as Swiss franc, euro, or pound sterling,
as well as, a monetary system as a whole, including the protocol,
the network, and all participants. To distinguish between the two,
the system \nomenclature{Bitcoin}{  } is written with a capital letter
B, and the monetary unit of account is \nomenclature{bitcoin}{ },
is written with a small letter b. Furthermore, there is the original
Bitcoin software, which is called\textbf{ \nomenclature{Bitcoin-Qt}{ }},
using this software package, or others, computer can connect to the
Bitcoin network.

\nomenclature{Bitcoin}{  }is built from the ground up to be decentralised
and anonymous. All software is open source and publicly available.
Anybody can perform any function within the network (user or node),
there is no central or essential node, there is redundancy in every
aspect.

The essential principle is that every user has a piece of software
on their computer, a\textbf{ \nomenclature{wallet}{ }}, which contains
a set of public addresses (like bank account numbers), each address
is mathematically linked with a \textbf{\nomenclature{private key}{ }}
(like a password). Using this secret key, users can create a digital
\nomenclature{signature}{ }, to prove that they are the owner of
an address. This signature, together with a transaction is sent to
a bitcoin \textbf{\nomenclature{miner}{ }}. The miner is a node in
the network, which verifies that the address and signature are linked,
after which the transaction is recorded in the public ledger, or \textbf{\nomenclature{blockchain}{ }}
and disseminated throughout the network.

There are a number of other features which are important to highlight.
Bitcoins are created through a process called mining, this is a computationally
intensive process used as the mechanism for transaction verification.
The Bitcoins created as a reward for mining becomes incrementally
smaller, until finally becoming zero. This is predicted to be around
the year 2140, and at that point around 21 million bitcoins will have
been created. There will never be more bitcoins in the system. Furthermore,
bitcoins will be lost if private keys are lost, these will never be
recovered. The system is thus strictly deflationary. To keep transactions
of every size possible, bitcoins are highly granular, every bitcoin
can be divided into a hundred million \textbf{\nomenclature{satoshi}{ }}
(named after the pseudonym of the creator).


\chapter{Why use Bitcoin?}

There are some key benefits to cryptocurrencies such as Bitcoin, a
few of which I will mention here.

Firstly, the system is self-regulating, meaning that no government
or institution controls it. For the user that means that their holdings
cannot be used as a policy instrument. Conventional fiat currencies
are owned by a government, which means that they can be used to e.g.
stimulate economic growth by printing extra money, which lessens the
value of individuals holdings.

Secondly, users control their own holdings on their own computer,
which eliminates the need for a bank. This brings the advantage that
users are not dependent on open hours, waiting lines, or e-banking
websites, it also means no fees. Additionally a user's holdings cannot
be wiped out by his or her bank bankrupting after speculative investing
or something of that sort.

Thirdly, transactions are as easy, instant, and costless as sending
an email. Bitcoin exists online digitally, which means that there
is no physical counterpart that needs to be moved, such as with gold,
or paper money. Additionally, it is equally valid across countries,
which means that there is no need to exchange it. This makes Bitcoin
an ideal mechanism for remittances, which nowadays can cost as much
as 10 percent. Additionally no government can apply capital restrictions
on these remittances.

Lastly, the instant nature of transactions means it is well suited
for online retailing. Credit card transactions seem instant, but in
fact are not. A significant percentage is in fact reversed after initial
clearance. This is very costly for retailers, this often occurs after
packaging of shipping has started.


\chapter{Is Bitcoin Legal?}

The short answer is yet, of course this does not mean that anything
that you use Bitcoin for is legal. That is to say, it is still illegal
to purchase elicit goods, or hide holdings from the tax bureau.


\chapter{Is Bitcoin money?}

What is money? Some answers are obvious, well established currencies
such as the US Dollar are. But the Euro is not a national currency,
it is not owned by a state. Zimbabwe owns their own currency, but
this has become essentially useless. What if money is moved from one,
bank account to another bank account at the same bank in the same
country, through ebanking. No physical change will occur (save for
a few bits on a computer driver somewhere in some remote datacentre).
In some places sea shells or other rare commodities such a gold or
ivory are accepted. Just as the number of possible moneys is alsmost
infiniate, so is the number of definitions, but a comptemporary (pre-Bitcoin)
consensus would dictate somethink like: 
\begin{quotation}
an object or entry in a record that is generally accepted as payment
for goods and services and debts in a certain socio-economic context
or country \textcolor{red}{CITE!!}.
\end{quotation}
We can observe a number of things from this definition. Firstly, money
does not have to be an object (such as a coin), but can also be an
entry in a record (such as a bank account balance). Secondly, it assumed
to be generally accepted (though not universally). Thirdly, a money
is not limited to only being accepted in a country, but can also be
accepted in a certain context. Lastly, the existance of one money
does not rule out the existance of others.

Bitcoin, the blockchain public ledger is a record. Generally accepted,
but generally accepted within a certain context. Also, coexcistance
of moneys.

The question of whether Bitcoin is money is a complicated question,
there are some standard which are used in academia to evaluate whether
something is money or not. Besides this there are the issues of legal
tender, and legal recognition. We will start by discussion the academic
criteria for a money, after which we will look at legal tender and
legal recognition. The standard criteria used in academia for analysing
money are:
\begin{enumerate}
\item medium of exchange
\item unit of account or numéraire
\item store of value
\end{enumerate}
These criteria can broadly be interpreted as follow. A medium of exchange

The criteria (1.) and (3.) are clearly fulfilled. Since cryptocoins
are virtual goods, and multiple copies can be made, there is no degeneration
over time. Furthermore, cryptocoins are a good medium of exchange,
transactions are quick and relatively cost less. The unit of account
is not a criterium that is currently well met. The volatility of cryptocurrencies
vis-à-vis fiat currencies makes it difficult to denominate goods or
services in these in cryptocoins. However, in that sense it is again
similar to gold, which also has a highly volatile price as denoted
in fiat currency.


\section{Legal Issues}

As mentioned above, there are two legal issues, recognition as being
legal, alongside a legal classification of Bitcoin, as well as the
issue of legal tender. We will discuss these here.


\subsection{Legal Recognition and Classification}

In the initial days of Bitcoin, especially when it was mainly used
for purchasing illicit goods, some held the belief that most governments
would perceive Bitcoin as a threat to their monopoly on currency and
for this reason would ban it. This already being contrasted by governments
of the EMU member states, who have volutarily surrendered their monetary
monopoly for a 'foreign' currency. It turned out most government where
more interested in taxing it. For this reason Bitcoin has in many
places been classified as a financial instument. If Bitcoins are derived
from mining, they are to be taxed using income tax. If Bitcoins are
purchased these will be taxed as financial holdings.


\subsection{Legal Tender}

What is legal tender? why do we need it, why dont we need it in bitcoin,
refer to the chapter of why bictoin


\part{The Economics}


\chapter{Bimetallism in Bitcoin and Litecoin?}

The situation of multiple currencies, especially in light of the gold
analogy, brings to mind the idea of bimetallism, the monetary system
where the currency is based not just on gold, but also on silver.
The two being interchangeable at a fixed rate.

Also in the light of bimetallism, we can consider Gresham's law. Bad
money drives out good. In the case of bimetallism, the most circulated
specie will be the one which is currently considered to be undervalued
by the official exchange rate. Meaning that if silver becomes more
abundant, its relative value will decrease, and people will first
spend their silver coins, before they spend their gold ones. Since
exchange rates between cryptocurrencies are freely floating, there
is no 'bad money', since each currency is valued at market rate. Interestingly
enough, the rate between Bitcoin and Litecoin seems to trade in a
very narrow band, especially considering the volatility vis-a-vis
fiat currencies. 

However, there is a crowding out of sort. This is perhaps best compared
to international currency competition. Where there is a competition
for the privilege of being a reserve currency. A foreign investor
can choose exchange his national fiat currency for e.g. the US Dollar
or the Euro. In these markets, the good currencies drive out the bad.


\chapter{Cryptocurrencies and Inflation}

As the value of bitcoin goes up, more people will choose to engage
in the lucrative mining. As extra computational power comes in, the
difficulty of mining automatically goes up, which means more computing
power is used for the same transactions. This is necessary to safeguard
the integrity of the network.

Having established the key features of the Bitcoin system, we will
now focus on two often heard economic concerns about the Bitcoin system.
Namely the issue of deflation and the issue of socially wasteful mining.

As mentioned above, there will only ever be around 21 million bitcoins,
and some will be lost, this makes the system deflationary, which is
troubling to economists. Deflation provides a disincentive to spend,
causing economic slowdown \citep[see e.g.][]{fisher1933debt}. Since
deflation causes the price of products to fall, it incentivises people
to save ans spend later. Additionally, these savings are not invested
properly, since deflation simultaneously provides a disincentive for
borrowers, by making future paybacks more expensive, raising the effective
interest rate.


\part{The Social Value}


\chapter{Is mining socially wasteful?}

The second issue is the value of the mining process. As described
above, the process of mining is the solving of computational problems,
in order to secure the integrity of the Bitcoin network. However,
the actual social value of the arithmetic solution itself is zero,
since it has no application other than Bitcoin integrity. As the number
of bitcoins is limited, and more money is invested in it, the value
can only increase. The increase in value will make it more lucrative
to engage in bitcoin mining, which means more computational power
will be devoted to this. To keep the system secure in face of this
extra computing power, the difficulty goes up. However, if the extra
power had not come in, this would not have been necessary. The extra
computers which engage in bitcoin mining thus do not add any social
value.

It has to be noted, that the bar for social value is being set very
high. Extraction, storage, and transfer of value is a resource intensive
enterprise. Consider the gold mining industry (note, this is the origin
of the term bitcoin mining). The extraction of gold is a very resource
intensive, dangerous, and pollutive process. After the extraction,
purification, and moulding, most gold is stored highly guarded in
vaults. Finally then, gold can be utilised to conduct transactions,
which involves shipping bars of gold across the ocean under maximum
security, only to be stored in another highly protected vault, upon
arrival \citep[see e.g.][]{friedman2008monetary}.

The solution for both these lies in the multiplicity of cryptocurrencies.
As mentioned above, every aspect of Bitcoin is open and publicly accessible,
it is therefore relatively easy to start an alternative Bitcoin, and
this has been done. There are in fact many currencies based on the
Bitcoin protocol, collectively referred to as cryptocurrencies. It
is relatively easy and cheap to construct a cryptocurrency, whereby
the number of available coins can be set by the creator. The most
popular alternative to Bitcoin is called Litecoin, its main difference
is that it processes transactions faster, and the total number of
coins is four times as high. Aside Litecoin, there are many other
alternative cryptocurrencies, most of which never gain momentum and
the value of which remains only trivially above zero. However, a significant
number does succeed. The momentum mechanism could perhaps best be
compared with a positive variation on currency attacks \citep{obstfeld1986rational,obstfeld1995logic,obstfeld1996models}.
Unlike the creation of new coins in an existing cryptocurrency, the
creation of new cryptocurrencies is relatively cheap.

The key point to observe here, is that cryptocurrencies do have value,
but only as a transaction mechanism. Hereby the biggest bottleneck
is probably the number of cryptocurrencies that merchants are willing
the accept simultaneously. However, due to their similarity, it is
very straightforward for e.g. merchants to accept multiple currencies.
When we combine the value with the relatively cost-less creation of
cryptocurrencies, we see an equilibrium in the number of cryptocurrencies
that is far higher than in the current situation.

Inflation is impossible within the Bitcoin network, as well as within
most other cryptocurrencies. However, inflation in the cryptocurrency
economy is still possible, through an expansion the number of cryptocurrencies.

It is also through this multiplicity that wasteful mining will be
limited. As noted above, cryptocurrencies have value, because they
are effective mechanisms for transactions. However, this is the only
source of value. If a cryptocurrency becomes too expensive (which
causes excessive mining), a new, lower valued, cryptocurrency will
arise. Since the value of this currency is lower, but it is equally
effective in transactions, value will flow out of the overvalued cryptocurrency
and into the undervalued one. This lowers the value of the overvalued
cryptocurrency, and less mining will be done here, reducing the waste.


\chapter{Conclusion}

In conclusion, cryptocurrencies such as Bitcoin have an enormous potential
as a transaction mechanism, giving users control of their own holdings
and making transactions instant and costs negligible. Two commonly
heard issues with cryptocurrencies are deflation and wasteful mining.
It can be shown that these issues are not pervasive, when the cryptocurrency
economy as a whole is considered. Cryptocurrencies derive their value
only from being an efficient transaction mechanism, if they appear
to become overvalued (causing excessive mining), other cryptocurrencies
will arise. This will increase the total number of cryptocoins (between
all cryptocurrencies), which will drive down the price, and limit
excessive mining.

\begin{center}
\printbibliography
\par\end{center}

\begin{center}
\printglossaries
\par\end{center}
\end{document}
