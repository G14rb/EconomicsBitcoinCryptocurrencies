
\chapter{Is Bitcoin Legal?}

The short answer is yet, of course this does not mean that anything
that you use Bitcoin for is legal. That is to say, it is still illegal
to purchase elicit goods, or hide holdings from the tax bureau.

The legal standing of Bitcoin is different in many countries. 


\section*{Currency or Financial Asset?}

In the initial days of Bitcoin, especially when it was mainly used
for purchasing illicit goods, some held the belief that most governments
would perceive Bitcoin as a threat to their monopoly on currency and
for this reason would ban it. This already being contrasted by governments
of the EMU member states, who have volutarily surrendered their monetary
monopoly for what is essentially a 'foreign' currency. It turns out
most government are more interested in taxing it, than they are in
outlawing it. For this reason Bitcoin has in many places been classified
as a financial instument. If Bitcoins are derived from mining, they
are to be taxed under income tax. If Bitcoins are purchased these
will be taxed as financial holdings. Again, the gold analogy holds.

In many places Bitcoin has been classified as a financial asset for
purposes of taxation. This means that it is not recognised as a currency
again for purposes of taxation. It several court rulings however,
it has been found that Bitcoin does act as a currency. In both cases
the legality of Bitcoin is underlines, however, the distinction can
be relevant in some cases. For instance, in case of a bankrupcy (such
as the 2014 Mt. Gox case) financial assets and currencies are treated
quite differently.
