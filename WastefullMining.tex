
\chapter{Is mining socially wasteful?}

The value of the mining process is an often raised issue. As described
above, the process of mining is the solving of computational problems,
in order to secure the integrity of the Bitcoin network. However,
the actual social value of the arithmetic solution itself is zero,
since it has no application other than Bitcoin integrity. As the number
of bitcoins is limited, and more money is invested in it, the value
can only increase. The increase in value will make it more lucrative
to engage in bitcoin mining, which means more computational power
will be devoted to this. To keep the system secure in face of this
extra computing power, the difficulty of verification automatically
goes up. However, had the extra power not come in, this would not
have been necessary. The extra computers which engage in bitcoin mining
thus do not add any social value.

It has to be noted, that the bar for social value is being set very
high. Extraction, storage, and transfer of value is a resource intensive
enterprise. Consider the gold mining industry (note, this is the origin
of the term bitcoin mining). The extraction of gold is a very resource
intensive, dangerous, and pollutive process. After the extraction,
purification, and moulding, most gold is stored highly guarded in
vaults. Finally then, gold can be utilised to conduct transactions,
which involves shipping bars of gold across the ocean under maximum
security, only to be stored in another highly protected vault, upon
arrival \citep[see e.g.][]{friedman2008monetary}.

The system of fiat currencies is already much more efficient than
this. There is no difficult extraction and purification process. Bank
notes are printed by governments and provides with authenticity checks.
Bank strike out debts against each other and only every so often is
printed money actually transfered from one bank to the other. However,
this is again a highly secured process of physically moving objects
(in this case banknotes) from one location to the other. Often followed
by a reverse transfer several days later.

The solution for both these lies in the multiplicity of cryptocurrencies.
As mentioned above, every aspect of Bitcoin is open and publicly accessible,
it is therefore relatively easy to start an alternative Bitcoin, and
this has been done. There are in fact many currencies based on the
Bitcoin protocol, collectively referred to as cryptocurrencies. It
is relatively easy and cheap to construct a cryptocurrency, whereby
the number of available coins can be set by the creator. The most
popular alternative to Bitcoin is called Litecoin, its main difference
is that it processes transactions faster, and the total number of
coins is four times as high. Aside Litecoin, there are many other
alternative cryptocurrencies, most of which never gain momentum and
the value of which remains only trivially above zero. However, a significant
number does succeed. Unlike the creation of new coins in an existing
cryptocurrency, the creation of new cryptocurrencies is relatively
cheap.

The key point to observe here, is that cryptocurrencies do have value,
but only as a transaction mechanism. Hereby the biggest bottleneck
is probably the number of cryptocurrencies that retailers are willing
the accept simultaneously. However, due to their similarity, it is
very straightforward for e.g. retailers to accept multiple currencies.
When we combine the value with the relatively costless creation of
cryptocurrencies, we see an equilibrium in the number of cryptocurrencies
that is far higher than in the current situation.

As noted earlier, cryptocurrencies have value, because they are effective
mechanisms for transactions. However, this is the only source from
which their value derives. If a cryptocurrency becomes too expensive
(which causes excessive mining), a new, lower valued, cryptocurrency
will arise. Since the value of this currency is lower, but it is equally
effective in transactions, value will flow out of the overvalued cryptocurrency
and into the undervalued one. This lowers the value of the overvalued
cryptocurrency, and less mining will be done here, reducing the waste.
