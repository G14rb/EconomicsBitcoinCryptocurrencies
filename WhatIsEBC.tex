
\chapter{What is Bitcoin?}

A short answer is that it is an online anonymous currency which is
not controlled by any single government or entity. However, from a
technical perspective there are many innovative features to Bitcoin,
and I will begin by highlighting the key ones. 

\nomenclature{Bitcoin}{ }is a confusing term. It is both a monetary
unit of denomination, such as Swiss franc, euro, or pound sterling,
as well as, a monetary system as a whole, including the protocol,
the network, and all participants. To distinguish between the two,
the system \nomenclature{Bitcoin}{  } is written with a capital letter
B, and the monetary unit of account is \nomenclature{bitcoin}{ },
is written with a small letter b. Furthermore, there is the original
Bitcoin software, which is called\textbf{ \nomenclature{Bitcoin-Qt}{ }},
using this software package, or others, computer can connect to the
Bitcoin network.

\nomenclature{Bitcoin}{  }is built from the ground up to be decentralised
and anonymous. All software is open source and publicly available.
Anybody can perform any function within the network (user or node),
there is no central or essential node, there is redundancy in every
aspect.


