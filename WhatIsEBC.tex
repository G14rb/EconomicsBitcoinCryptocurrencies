
\chapter{What is Bitcoin?}

A short answer is that it is an online anonymous currency which is
not controlled by any single government or entity. However, from a
technical perspective there are many innovative features to Bitcoin,
and I will begin by highlighting the key ones. 

\nomenclature{Bitcoin}{ }is a confusing term. It is both a monetary
unit of denomination, such as Swiss franc, euro, or pound sterling,
as well as, a monetary system as a whole, including the protocol,
the network, and all participants. To distinguish between the two,
the system \nomenclature{Bitcoin}{  } is written with a capital letter
B, and the monetary unit of account is \nomenclature{bitcoin}{ },
is written with a small letter b. Furthermore, there is the original
Bitcoin software, which is called\textbf{ \nomenclature{Bitcoin-Qt}{ }},
using this software package, or others, computer can connect to the
Bitcoin network.

\nomenclature{Bitcoin}{  }is built from the ground up to be decentralised
and anonymous. All software is open source and publicly available.
Anybody can perform any function within the network (user or node),
there is no central or essential node, there is redundancy in every
aspect.

The essential principle is that every user has a piece of software
on their computer, a\textbf{ \nomenclature{wallet}{ }}, which contains
a set of public addresses (like bank account numbers), each address
is mathematically linked with a \textbf{\nomenclature{private key}{ }}
(like a password). Using this secret key, users can create a digital
\nomenclature{signature}{ }, to prove that they are the owner of
an address. This signature, together with a transaction is sent to
a bitcoin \textbf{\nomenclature{miner}{ }}. The miner is a node in
the network, which verifies that the address and signature are linked,
after which the transaction is recorded in the public ledger, or \textbf{\nomenclature{blockchain}{ }}
and disseminated throughout the network.

There are a number of other features which are important to highlight.
Bitcoins are created through a process called mining, this is a computationally
intensive process used as the mechanism for transaction verification.
The Bitcoins created as a reward for mining becomes incrementally
smaller, until finally becoming zero. This is predicted to be around
the year 2140, and at that point around 21 million bitcoins will have
been created. There will never be more bitcoins in the system. Furthermore,
bitcoins will be lost if private keys are lost, these will never be
recovered. The system is thus strictly deflationary. To keep transactions
of every size possible, bitcoins are highly granular, every bitcoin
can be divided into a hundred million \textbf{\nomenclature{satoshi}{ }}
(named after the pseudonym of the creator).
